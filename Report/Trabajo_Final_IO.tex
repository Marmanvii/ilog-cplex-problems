\documentclass[a4paper,12pt]{article} 
\usepackage[top = 2.5cm, bottom = 2.5cm, left = 2.5cm, right = 2.5cm]{geometry} 
\usepackage[T1]{fontenc}
\usepackage{multirow} 
\usepackage{amsmath}
\usepackage{booktabs} 
\usepackage{graphicx} 
\usepackage{setspace}
\usepackage{float}
\usepackage{fancyhdr}

\pagestyle{fancy}

\fancyhf{}

\lhead{\footnotesize Inv. de Operaciones: Tarea Final}

\rhead{\footnotesize Martin Mancilla, Claudio Duran}

\cfoot{\footnotesize \thepage} 

\begin{document}
	
\thispagestyle{empty}

\begin{tabular}{p{15.5cm}}
	{\large \bf Investigación de Operaciones} \\
	Universidad Católica del Maule \\ Martín Mancilla V. - Claudio Durán N.\\
	\textit{19.386.399-k} - \textit{19.215.697-1} \\
	\hline
	\\
\end{tabular}

\vspace*{0.3cm}

\begin{center}
	{\Large \bf Trabajo Final} 
	\vspace{2mm}
	
	% YOUR NAMES GO HERE
	{Resolución de Problemas de Optimización}
	
\end{center}  

\vspace{0.4cm}

\section{Primer Problema}
\subsection{Formule  el  modelo  que  permita  obtener  el  portafolio de  inversión  que  optimice  el  retorno  esperado  de  la	corporación y simultáneamente no viole su política de inversión.}
\subsubsection{Variable de Decisión}
\vspace{0.4cm}
\begin{equation*}
	\begin{split}
		N &= (1,2,3,4,5) \\
		X_i & = \text{Cantidad invertida en categoría } i \text{ de inversión. } \forall i \in N
	\end{split}
\end{equation*}



	
\end{document}